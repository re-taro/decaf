\documentclass[11pt,twocolumn]{classes/yokou}

\title{静的型付けの意味における健全性に焦点を当てた TypeScript 型検査機}
\etitle{A TypeScript type checker with focus on soundness in the context of static typing}
\author{糸川 倫太朗}
\eauthor{ITOKAWA Rintaro}
\laboratory{青山研究室}

\begin{document}

\maketitle

\section{背景}

\subsection{TypeScript}

TypeScript は,Microsoft によって開発された,ECMAScript 5 を拡張したプログラミング言語である.
TypeScript の特徴として,型アノテーションや型エイリアス,関数オーバーロードのサポートが挙げられる.
これにより,静的型検査を通じてプログラムの誤りを検出できる.
一方で,\texttt{any} 型の存在により,型チェックを回避することも可能である.
\texttt{any} 型は,型アノテーションを省略可能にし,JavaScript からの移行を容易にする役割を果たす.
この柔軟性により,TypeScript は型検査の範囲を段階的に広げつつ利用できるデザインとなっている.

\subsection{Gradual Typing}

Gradual Typing 1) は,Siek と Taha によって 2006 年に提案された,静的型付けと動的型付けを融合させる手法である.
この手法では,プログラム内で静的検査を適用する箇所をプログラマが選択できる\texttt{?} 型が導入され,静的検査を回避するためのアノテーションとして機能する点で,TypeScript の \texttt{any} 型に類似している.

\subsection{Criteria for Gradual Typing}

Gradual typing という概念は Siek ら 2) の 2015 年の論文で整理され,以下の 5 つの条件を満たすべきと定義された.2006 年のオリジナル体系はこれらの条件を全て満たしている.

\begin{enumerate}
	\item 静的型付けの内包: 型アノテーションが完全に付与されたプログラムは,通常の静的型付けシステムと同じ挙動を示す.
	\item 動的型付けの内包: 型アノテーションが全て\texttt{?}型であるプログラムは,動的型付けシステムと同じ挙動を示す.
	\item 健全性: 型エラーは必ずランタイムで検出可能である.
	\item Blame-Subtyping Theorem: $T_1 <: T_2$ であれば,$T_1$ から $T_2$ へのキャストはランタイム型エラーを引き起こさない.
	\item Gradual Guarantee: 静的型検査に合格したプログラムの型アノテーションを減らしても検査は成功し,動作も変わらない.
\end{enumerate}

これらの条件により,静的型付けと動的型付けを柔軟に組み合わせるシステムが明確に定義されている.

\subsection{TypeScriptとCriteria for Gradual Typing}

TypeScript の型システムを Siek らの基準に照らして検討したところ,TypeScript は gradual typing の意味での健全性を失っており,ランタイム型チェックを行わない言語設計がその要因であることが明らかとなった.

\subsection{Safe TypeScriptとその課題}

Safe TypeScript は,TypeScript の gradual typing における健全性の欠如に対応するため Microsoft Research によって開発された.
Safe TypeScript ではランタイム型チェックのオーバーヘッド削減が試みられ,約 15\% に抑えられたが,それでも実用性に問題がある.
特に現代の Web 開発では,ランタイムの負荷がユーザビリティに大きな影響を与えるため,Safe TypeScript は現実的な選択肢とは言えない.

\section{目的}

TypeScript の柔軟性と静的型検査による安全性を両立させることを目的とし,新しい型検査ツール「decaf」を提案する.
これは,開発者が求める「省略された型アノテーションを補完した上での静的型検査」という柔軟性を実現し,ランタイムのオーバーヘッドを削減しつつ,健全性を保つことを目指している.

\section{decaf の特徴}

decaf は,TypeScript エコシステムに新たなアプローチを導入するコンパイラである.
その特徴は以下の手順に基づく型検査プロセスにある.
まず,入力ファイルをパースし AST を生成する.この AST は,既存の ESTree や swc の形式とは異なり,decaf 専用に設計されている.この AST は,型検査機が効率的に動作するよう変換され,生成された構造は「TypeID」と呼ばれる.これにより,モダンな型検査アルゴリズムの適用が可能となる.
型検査では,プログラム内の値や構造を型として,関数やブロックの振る舞いをイベントとしてエンコードする.decaf は型を「その値が取りうる可能性の集合」として扱い,AST の走査中に型の集合を広げることで,あらゆる実行時の可能性を型チェック時に考慮する.この手法により,条件分岐やループにおける動的な型の変化にも柔軟に対応する.
さらに,従来の TypeScript コンパイラとの主な違いとして,Literal widening を行わない点が挙げられる.これは,リテラル型が意図せず集合型にアップキャストされる問題を防ぎ,より正確な型推論を可能にする.
また,型アノテーションが省略された場合でも,その型は暗黙的に\texttt{any}とならず,適切な型が推定される.

\section{実験}

decaf の性能を評価する.
decaf は tsc よりも厳格な型システムを提供するため,型検査に多くの経路を処理する.
このため,decaf の型検査時間が tsc より長くなることが予想される.
評価には,350 件の構文テストスイートから生成したソースコードを用い,decaf と tsc での型検査結果を比較した.

\section{結果と考察}

各テストケースにおける decaf と tsc の平均の型検査時間を比較した結果を表\ref{tab:comparison} に示す.
数値の単位はミリ秒である.
列見出しはそれぞれのテストケースの繰り返し回数を示している.

\begin{table}[h]
	\caption{decaf と tsc の平均の型検査時間比較}
	\begin{tabular}{|c|c|c|c|c|}
		\hline
		\textbf{n}     & 1      & 10     & 20     & 40     \\ \hline
		\textbf{decaf} & 19.47  & 162.54 & 484.79 & 1610   \\ \hline
		\textbf{tsc}   & 272.78 & 323.83 & 371.18 & 451.96 \\ \hline
	\end{tabular}
	\label{tab:comparison}
\end{table}

表\ref{tab:comparison}に示す通り,tsc の型検査速度は平均 272.78ms〜451.96ms の範囲であり,計算量はプログラムのサイズ$n$に対して$O(n)$の傾向を示している($y = 4.52n + 275$, $R^2 = 0.993$).
一方,decaf では,の型検査速度が 19.47ms〜1,610.00ms と幅広く,計算量は$O(n^2)$となっている($y = 0.799n^2 + 7.58n + 8.68$, $R^2 = 1.00$).
回帰式から decaf の型検査速度は tsc よりも遅い場合があるが,それは decaf の型検査速度はプログラムのサイズに対してより高次の計算量を持つことが原因である.
これは,decaf がより厳格な型システムを提供するため,型検査に多くの経路を処理するためである.
また y 切片はそれぞれのコンパイラの初期化時間を表しており,decaf は tsc よりも初期化時間が短いことがわかる.
そのため,simple.tsx のような 2000 行程度の小規模なプログラムにおいては,decaf は tsc よりも高速に型検査でき,一般的なケースで decaf が優位性を持つことが示唆される.

\section{まとめ}

型検査を通して健全性を保証するための TypeScript 型検査機を実装した.
プログラムの挙動を静的に解析できる型システムを構築したことで,ランタイムエラーを事前に検出できるようになった.
TypeScript では静的解析の際,検知できずにランタイムエラーとして現れる問題を,decaf では静的解析時に検知できるようになった.

\begin{thebibliography}{9}
	\bibitem{} Robin Milner. A theory of type polymorphism in programming. Journal of Computer and System Sciences, Vol. 17, No. 3, pp. 348-375, 19
	\bibitem{} Jeremy G. Siek, Michael M. Vitousek, Matteo Cimini, and John Tang Boyland. Re-fined Criteria for Gradual Typing. In Thomas Ball, Rastislav Bodík, Shriram Krishna-murthi, Benjamin S. Lerner, and Greg Morriset, editors, 1st Summit on Advances in Programming Languages (SNAPL 2015), Vol. 32 of Leibniz International Proceedings in Informatics (LIPIcs), pp. 274-293, Dagstuhl, Germany, 2015. Schloss Dagstuhl - Leibniz-Zentrum für Informatik.
\end{thebibliography}

\end{document}
