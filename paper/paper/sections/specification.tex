\chapter{型検査の仕様}

それぞれの機能に対する型検査の仕様を示す.例として,decaf と TypeScript における型検査の結果を示す.

\section{変数の宣言}

変数名の後にコロン(:)を付け,その後に型を指定することで変数の型を指定できる.
変数の型が指定されていない場合,その変数は \texttt{any} 型として扱われる.
\ref{lst:declaration} のコードを例にすると,decaf では 2 は \texttt{2} 型であり,\texttt{number} 型の部分集合であるため,\texttt{x} は \texttt{number} 型として扱われる.
しかし,\texttt{y} は \texttt{string} 型であるため,エラーが発生する.
同様に,\texttt{z} は \texttt{object} 型であるため,エラーが発生する.
diagnostics は\ref{lst:diagnostics:declaration:decaf} に示す.

\begin{lstlisting}[caption=変数宣言の例, label=lst:declaration]
  const x: number = 2
  const y: string = 2
  const z: object = 4
\end{lstlisting}

\begin{minipage}{0.45\textwidth}
    \begin{lstlisting}[caption=decaf の diagnostics, label=lst:diagnostics:declaration:decaf]
    Type 2 is not assignable to type string
    Type 4 is not assignable to type object
  \end{lstlisting}
\end{minipage}
\hfill
\begin{minipage}{0.45\textwidth}
    \begin{lstlisting}[caption=tsc の diagnostics, label=lst:diagnostics:declaration:tsc]
    Type 'number' is not assignable to type 'string'.
    Type 'number' is not assignable to type 'object'.
  \end{lstlisting}
\end{minipage}

\section{変数への代入}

変数に代入する値の型が変数の型と一致しない場合,エラーが発生する.
\ref{lst:assignment} のコードを例にすると,decaf では \texttt{x} は \texttt{number} 型であるため,\texttt{"hello world"} は \texttt{number} 型として扱われる.
しかし,\texttt{number} 型に \texttt{"hello world"} 型を代入できないため,エラーが発生する.
diagnostics は\ref{lst:diagnostics:assignment:decaf} に示す.

\begin{lstlisting}[caption=変数への代入の例, label=lst:assignment]
  let x: number = 3
  x = "hello world"
\end{lstlisting}

\begin{minipage}{0.45\textwidth}
    \begin{lstlisting}[caption=decaf の diagnostics, label=lst:diagnostics:assignment:decaf]
    Type "hello world" is not assignable to type number
  \end{lstlisting}
\end{minipage}
\hfill
\begin{minipage}{0.45\textwidth}
    \begin{lstlisting}[caption=tsc の diagnostics, label=lst:diagnostics:assignment:tsc]
    Type 'string' is not assignable to type 'number'.
  \end{lstlisting}
\end{minipage}

\section{変数への参照}

変数への参照の型が変数の型と一致しない場合,エラーが発生する.
\ref{lst:reference} のコードを例にすると,decaf では \texttt{a} は \texttt{3} 型であるため,\texttt{b} は \texttt{3} 型として扱われる.
しかし,\texttt{3} 型に \texttt{string} 型を代入できないため,エラーが発生する.
diagnostics は\ref{lst:diagnostics:reference:decaf} に示す.

\begin{lstlisting}[caption=変数への参照の例, label=lst:reference]
  const a = 3
  const b: string = a
\end{lstlisting}

\begin{minipage}{0.45\textwidth}
    \begin{lstlisting}[caption=decaf の diagnostics, label=lst:diagnostics:reference:decaf]
    Type 3 is not assignable to type string
  \end{lstlisting}
\end{minipage}
\hfill
\begin{minipage}{0.45\textwidth}
    \begin{lstlisting}[caption=tsc の diagnostics, label=lst:diagnostics:reference:tsc]
    Type 'number' is not assignable to type 'string'.
  \end{lstlisting}
\end{minipage}

\section{変数への再代入}

変数への再代入の型が変数の型と一致しない場合,エラーが発生する.
\ref{lst:reassignment} のコードを例にすると,decaf では \texttt{a} は \texttt{2} 型であるため,\texttt{"hello world"} は \texttt{2} 型として扱われる.
しかし,\texttt{2} 型に \texttt{"hello world"} 型を最代入できないため,エラーが発生する.
diagnostics は\ref{lst:diagnostics:reassignment:decaf} に示す.

\begin{lstlisting}[caption=変数への再代入の例, label=lst:reassignment]
  let a = 2
  a = "hello world"
  a satisfies number
\end{lstlisting}

\begin{minipage}{0.45\textwidth}
    \begin{lstlisting}[caption=decaf の diagnostics, label=lst:diagnostics:reassignment:decaf]
    Expected number, found "hello world"
  \end{lstlisting}
\end{minipage}
\hfill
\begin{minipage}{0.45\textwidth}
    \begin{lstlisting}[caption=tsc の diagnostics, label=lst:diagnostics:reassignment:tsc]
    Type 'string' is not assignable to type 'number'.
  \end{lstlisting}
\end{minipage}

\section{存在しない変数への参照}

存在しない変数への参照がある場合,エラーが発生する.
\ref{lst:nexists} のコードを例にすると,\texttt{nexists} は存在しないため,エラーが発生する.
diagnostics は\ref{lst:diagnostics:nexists:decaf} に示す.

\begin{lstlisting}[caption=存在しない変数への参照の例, label=lst:nexists]
  const exists = 2;
  nexists;
\end{lstlisting}

\begin{minipage}{0.45\textwidth}
    \begin{lstlisting}[caption=decaf の diagnostics, label=lst:diagnostics:nexists:decaf]
    Could not find variable 'nexists' in scope
  \end{lstlisting}
\end{minipage}
\hfill
\begin{minipage}{0.45\textwidth}
    \begin{lstlisting}[caption=tsc の diagnostics, label=lst:diagnostics:nexists:tsc]
    Cannot find name 'nexists'. Did you mean 'exists'?
  \end{lstlisting}
\end{minipage}

\section{変数の宣言前の代入}

変数の宣言前に代入がある場合,エラーが発生する.
\ref{lst:predeclaration} のコードを例にすると,decaf では \texttt{a} は宣言される前に代入されているため,エラーが発生する.

\begin{lstlisting}[caption=変数の宣言前の代入の例, label=lst:predeclaration]
  a = 3;
  let a = 2;
\end{lstlisting}

\begin{minipage}{0.45\textwidth}
    \begin{lstlisting}[caption=decaf の diagnostics, label=lst:diagnostics:predeclaration:decaf]
    Cannot assign to 'a' before declaration
  \end{lstlisting}
\end{minipage}
\hfill
\begin{minipage}{0.45\textwidth}
    \begin{lstlisting}[caption=tsc の diagnostics, label=lst:diagnostics:predeclaration:tsc]
    Block-scoped variable 'a' used before its declaration.
  \end{lstlisting}
\end{minipage}

\section{存在しない変数への代入}

存在しない変数への代入がある場合,エラーが発生する.
\ref{lst:doesNotExist} のコードを例にすると,\texttt{doesNotExist} は存在しないため,エラーが発生する.
diagnostics は\ref{lst:diagnostics:doesNotExist:decaf} に示す.

\begin{lstlisting}[caption=存在しない変数への代入の例, label=lst:doesNotExist]
  doesNotExist = 3;
\end{lstlisting}

\begin{minipage}{0.45\textwidth}
    \begin{lstlisting}[caption=decaf の diagnostics, label=lst:diagnostics:doesNotExist:decaf]
    Could not find variable 'nexists' in scope
  \end{lstlisting}
\end{minipage}
\hfill
\begin{minipage}{0.45\textwidth}
    \begin{lstlisting}[caption=tsc の diagnostics, label=lst:diagnostics:doesNotExist:tsc]
    Cannot find name 'nexists'.
  \end{lstlisting}
\end{minipage}

\section{重複して宣言された変数}

変数が重複して宣言された場合,エラーが発生する.
\ref{lst:duplication} のコードを例にすると,decaf では \texttt{a} は 2 で宣言されているため,\texttt{a} は 2 である.
しかし,\texttt{a} は 3 で再宣言されているため,エラーが発生する.
ここで,\texttt{\{ \}} で囲まれた部分はスコープを示しているため,エラーにならない.
diagnostics は\ref{lst:diagnostics:duplication:decaf} に示す.

\begin{lstlisting}[caption=重複して宣言された変数の例, label=lst:duplication]
  const a = 2
  {
  	const a = 3;
  	a satisfies 3;
  }
  a satisfies 2;
  const a = 3;
\end{lstlisting}

\begin{minipage}{0.45\textwidth}
    \begin{lstlisting}[caption=decaf の diagnostics, label=lst:diagnostics:duplication:decaf]
    Cannot redeclare variable 'a'
  \end{lstlisting}
\end{minipage}
\hfill
\begin{minipage}{0.45\textwidth}
    \begin{lstlisting}[caption=tsc の diagnostics, label=lst:diagnostics:duplication:tsc]
    Cannot redeclare block-scoped variable 'a'.
    Cannot redeclare block-scoped variable 'a'.
  \end{lstlisting}
\end{minipage}

\section{変数のシャドーイング}

変数のシャドーイングがある場合,エラーが発生する.
\ref{lst:shadowing} のコードを例にすると,decaf では \texttt{a} は 2 で宣言されているため,\texttt{a} は 2 である.
\texttt{\{ \}} で囲まれた部分はスコープを示しているため,\texttt{a} は 3 である.
しかし,\texttt{satisfies}で \texttt{a} が 2 であることを示しているため,エラーが発生する.
diagnostics は\ref{lst:diagnostics:shadowing:decaf} に示す.

\begin{lstlisting}[caption=変数のシャドーイングの例, label=lst:shadowing]
  const a = 2
  {
  	const a = 3;
  	a satisfies 2;
  }
\end{lstlisting}

\begin{minipage}{0.45\textwidth}
    \begin{lstlisting}[caption=decaf の diagnostics, label=lst:diagnostics:shadowing:decaf]
    Expected 2, found 3
  \end{lstlisting}
\end{minipage}
\hfill
\begin{minipage}{0.45\textwidth}
    \begin{lstlisting}[caption=tsc の diagnostics, label=lst:diagnostics:shadowing:tsc]
    Type '3' does not satisfy the expected type '2'.
  \end{lstlisting}
\end{minipage}

\section{未初期化の変数}

未初期化の変数は \texttt{undefined} 型として扱われる.
未初期化の変数に型が指定されている場合,エラーが発生する.
\ref{lst:uninitialized} のコードを例にすると,decaf では \texttt{b} は未初期化のため,\texttt{undefined} 型として扱われる.
しかし,\texttt{undefined} 型に \texttt{string} 型を期待できないため,エラーが発生する.
diagnostics は\ref{lst:diagnostics:uninitialized:decaf} に示す.

\begin{lstlisting}[caption=未初期化の変数の例, label=lst:uninitialized]
  let b;
  b satisfies string;
\end{lstlisting}

\begin{minipage}{0.45\textwidth}
    \begin{lstlisting}[caption=decaf の diagnostics, label=lst:diagnostics:uninitialized:decaf]
    Expected string, found undefined
  \end{lstlisting}
\end{minipage}
\hfill
\begin{minipage}{0.45\textwidth}
    \begin{lstlisting}[caption=tsc の diagnostics, label=lst:diagnostics:uninitialized:tsc]
    // エラーにならない
  \end{lstlisting}
\end{minipage}

\section{存在しないプロパティへの参照}

存在しないプロパティへの参照がある場合,エラーが発生する.
\ref{lst:nonexistent} のコードを例にすると,decaf では \texttt{my\_obj} は \texttt{\{ a: 3 \}} 型であるため,\texttt{a} は \texttt{3} 型として扱われる.
しかし,\texttt{b} は存在しないため,エラーが発生する.
diagnostics は\ref{lst:diagnostics:nonexistent:decaf} に示す.

\begin{lstlisting}[caption=存在しないプロパティの例, label=lst:nonexistent]
  let my_obj = { a: 3 }
  const a = my_obj.a
  const b = my_obj.b
\end{lstlisting}

\begin{minipage}{0.45\textwidth}
    \begin{lstlisting}[caption=decaf の diagnostics, label=lst:diagnostics:nonexistent:decaf]
    No property 'b' on { a: 3 }
  \end{lstlisting}
\end{minipage}
\hfill
\begin{minipage}{0.45\textwidth}
    \begin{lstlisting}[caption=tsc の diagnostics, label=lst:diagnostics:nonexistent:tsc]
    Property 'b' does not exist on type '{ a: number; }'.
  \end{lstlisting}
\end{minipage}
